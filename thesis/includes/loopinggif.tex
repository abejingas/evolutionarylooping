\chapter{Looping GIFs}
Das Graphics Interchange Format (GIF) ist ein Bildformatstandard, der mit einer Farbpalette und einer verlustfreien Kompression arbeitet. Eine spezielle Eigenschaft dieses Bildformates ist, dass es mehrere Einzelbilder in einer Datei abspeichern kann, die von geeigneten Bildbeobachtungsprogrammen sowie den meisten Webbrowsern als Animationen interpretiert werden. Dabei wird die Animation meist in einer Endlosschleife abgespielt: Wird das Ende erreicht, so wird die Animation ohne Pause wieder von vorne abgespielt. Zwar wurden im GIF gespeicherte Bilder im Internet immer mehr durch das neue PNG-Format abgelöst, erlebte jedoch ein Wiederauferstehen in Bilder-Foren wie \textit{tumblr} \cite{Tumblr:1}, \textit{4chan} \cite{4chan:1} oder \textit{Imgur} \cite{Imgur:1}. 

Eine spezielle Gattung unter den GIF-Animationen in Internetforen stellen die sogenannten `"Looping GIFs"' dar. Diese Animationen haben die spezielle Eigenschaft, dass sich ihr erstes und letztes Bild so ähnlich sind, dass der Übergang beim Abspielen der Animation nicht auffällt. Diese Gattung erfreut sich großer Beliebtheit, da die Benutzer der Foren es als eine Herausforderung ansehen, den Schnitt in der Animation zu finden. Meist sind diese Looping GIFs etwa 1 bis 10 Sekunden lange Ausschnitte aus bekannten Videos oder Filmen. Diese Ausschnitte werden von Benutzern in den Foren meistens durch das mehrfache exakte Betrachten des Videomaterials mit dem Auge ermittelt - eine zeitaufwändige und anstrengende Arbeit. Jedoch handelt es sich hierbei um eine Aufgabe, die sehr gut durch einen Algorithmus übernommen werden kann. Ein unter dem Pseudonym `"Zulko"' \cite{Zulko:1} bekannter Entwickler hat sich dieser Aufgabe gewidmet und einen iterativen Algorithmus \cite{Goldberg:1} zur Suche von Looping GIFs in Videodateien entworfen \cite{Zulko:2}. 