\chapter{Schlussbemerkungen}
\label{cha:Schluss}

\begin{itemize}
	\item[$\Box$] \textbf{Titelseite:} Länge des Titels (Zeilenumbrüche), Name, Studiengang, Datum.
	\item[$\Box$] \textbf{Erklärung:} vollständig Unterschrift.
	\item[$\Box$] \textbf{Inhaltsverzeichnis:} balanzierte Struktur, Tiefe, Länge der Überschriften.
	\item[$\Box$] \textbf{Kurzfassung/Abstract:} präzise Zusammenfassung, Länge, gleiche Inhalte.
	\item[$\Box$] \textbf{Überschriften:} Länge, Stil, Aussagekraft.
	\item[$\Box$] \textbf{Typographie:} sauberes Schriftbild, keine "`manuellen"' Abstände zwischen Absätzen oder Einrückungen, keine überlangen Zeilen, Hervorhebungen, Schriftgröße, Platzierung von Fußnoten.
	\item[$\Box$] \textbf{Punktuation:} Binde- und Gedankenstriche richtig gesetzt, Abstände nach Punkten (\va\ nach Abküzungen).
	\item[$\Box$] \textbf{Abbildungen:} Qualität der Grafiken und Bilder, Schriftgröße und -typ in Abbildungen, Platzierung von Abbildungen und Tabellen, Captions. Sind \emph{alle} Abbildungen (und Tabellen) im Text referenziert?
	\item[$\Box$] \textbf{Gleichungen/Formeln:} mathem.\ Elemente auch im Fließtext richtig gesetzt, explizite Gleichungen richtig verwendet, Verwendung von mathem.\ Symbolen.
	\item[$\Box$] \textbf{Quellenangaben:} Zitate richtig referenziert, Seiten- oder Kapitelangaben.
	\item[$\Box$] \textbf{Literaturverzeichnis:} mehrfach zitierte Quellen nur einmal angeführt, Art der Publikation muss in jedem Fall klar sein, konsistente Einträge, Online-Quellen (URLs) sauber angeführt.
	\item[$\Box$] \textbf{Sonstiges:} ungültige Querverweise (\textbf{??}), Anhang, Papiergröße der PDF-Datei (A4 = $8.27 \times 11.69$ Zoll), Druckgröße und -qualität.
\end{itemize}


